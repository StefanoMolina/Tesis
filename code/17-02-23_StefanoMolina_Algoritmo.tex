\documentclass[letterpaper]{paper}

%\usepackage{algorithm2e}
%\usepackage[linesnumbered]{algorithm2e}
%\usepackage[linesnumbered,ruled]{algorithm2e}
\usepackage[linesnumbered,ruled,vlined]{algorithm2e}
\usepackage{amsmath}
\usepackage{amssymb}
\usepackage{amsthm}
\usepackage[spanish]{babel}
\usepackage{epsfig}
\usepackage{graphicx}

\newcommand{\bolde}{\boldsymbol{e}}
\newcommand{\boldA}{\boldsymbol{A}}
\newcommand{\boldy}{\boldsymbol{y}}
\newcommand{\boldY}{\boldsymbol{Y}}
\newcommand{\boldf}{\boldsymbol{f}}
\newcommand{\boldF}{\boldsymbol{F}}
\newcommand{\boldb}{\boldsymbol{b}}
\newcommand{\boldB}{\boldsymbol{B}}
\newcommand{\boldbeta}{\boldsymbol{\beta}}
\newcommand{\boldBeta}{\boldsymbol{\Beta}}
\newcommand{\boldsigma}{\boldsymbol{\sigma}}
\newcommand{\boldSigma}{\boldsymbol{\Sigma}}
\newcommand{\boldzero}{\boldsymbol{0}}
\newcommand{\boldI}{\boldsymbol{I}}

\newcommand{\NormD}{\text{N}}
\newcommand{\Real}{\mathbb{R}}
\newcommand{\diag}{\text{diag}}
\newcommand{\etr}{\text{etr}}
\newcommand{\tr}{\text{tr}}

\title{Bayesian Factor Model}
\author{J. C. Mart\'inez-Ovando}

\begin{document}
	
\maketitle

\section{Estructura y especificaci\'on}

Suponemos que los datos $\{\boldy_1,\ldots,\boldy_T\}$ toman valores en $\Real^{p}$ (con $p<\infty$). 

Para cualquier numero $k\leq p$ se tiene que el modelo de factores queda especificado como
\begin{eqnarray}
	\boldy_t|\boldf_t 
	  & \sim &
	  \NormD\left(\boldy_t|\boldB\boldf_t,\boldSigma\right)
	  \nonumber \\
	\boldf_t
	  & \sim &
	  \NormD\left(\boldf_t|\boldzero,\boldI\right),
	  \nonumber 
\end{eqnarray}
para $t=1,\ldots,T$, donde 
\begin{itemize}
  \item[$\boldf_t$] es el vector de factores $k$-dimensional asociado con $\boldy_t$  
  \item[$\boldB$] es una matriz de dimensi\'on $(p\times k)$ con los vectores de cargas asociados con los factores $\boldf_t$ para $\boldy_t$
  \item[$\boldSigma$] es una matriz $\diag\{\sigma_1^{2},\ldots,\sigma_p^{2}\}$ positivo definida de dimensi\'on $(p\times p)$
  \item[$\boldI$] es la matriz identidad de dimensi\'on $(k\times k)$.
\end{itemize}
De la especificaci\'on anterior se sigue que la varianza no condicional de las observaciones es,
\begin{eqnarray}
 var(\boldy_t|\boldB,\boldSigma)
   & = &
   \boldB\boldB' + \boldSigma.
   \nonumber
\end{eqnarray}

El modelo implica que, condicional a los factores comunes, las variables observables no están correlacionadas: por lo tanto, los factores comunes explican toda la estructura de dependencia entre las $p$ variables. lo anterior motiva para considerar:
\begin{itemize}
	\item la sucesi\'on de factores $\{\boldf_t\}_{t=1}^{T}$ como un conjunto de variables latentes, y
	\item al conjunto $\{\boldB,\boldSigma\}$ como los par\'ametros del modelo.
\end{itemize}

La verosimilitud extendida del modelo (i.e. verosimilitud para par\'ametros y variables latentes), se define como
\begin{eqnarray}
lik\left(\boldB,\boldSigma,\{\boldf_t\}_{t=1}^{T}|\{\boldy_t\}_{t=1}^{T}\right)
	& = &
	p\left(\{\boldy_t\}_{t=1}^{T}|\boldB,\boldSigma,\{\boldf_t\}_{t=1}^{T}\right)
	\nonumber \\
	& \propto &
	\det(\boldSigma)^{-T/2}
	\etr\left\{-(1/2)\boldSigma^{-1}\bolde\bolde'\right\},
	\nonumber
\end{eqnarray}
donde $\etr\{\boldA\}=\exp\{\tr(A)\}$ para toda matriz $\boldA$.

\subsection{Distribuci\'on inicial}

\subsection{Distribuci\'on final}
	
\begin{algorithm}[H]
	\SetAlgoLined 
	Inicializaci\'on: $\boldB^{(0)}$, $\boldSigma^{(0)}$, Y $\{\boldf^{(0)}_t\}_{t=1}^{T}$\\
	\For{$k$ en $1:M$}{
		simular $\boldB^{(k)}$ de 
		$$
		p\left(\boldB|\boldSigma^{(k-1)},\{\boldf^{(k-1)}_t\}_{t=1}^{T},\{\boldy_t\}_{t=1}^{T}\right)
		$$\\
		simular $\boldSigma^{(k)}$ de
		$$
		p\left(\boldSigma|\boldB^{(k)},\{\boldf^{(k-1)}_t\}_{t=1}^{T},\{\boldy_t\}_{t=1}^{T}\right)
		$$\\
		simular $\{\boldf^{(k)}_t\}_{t=1}^{T}$ de
		$$
		p\left(\{\boldf_t\}_{t=1}^{T}|\boldB^{(k)},\boldSigma^{(k)},\{\boldy_t\}_{t=1}^{T}\right)
		$$
	}
	\Return{$\left\{\boldB^{(k)},\boldSigma^{(k)},\{\boldf^{(k)}_t\}_{t=1}^{T}\right\}_{k=1}^{M}$}
	\caption{Gibbs sampler para el modelo de factores}
\end{algorithm}

\end{document}